\section{Conclusion}

In recent times global awareness of the importance of data privacy has grown. The IT companies are reviewing their policies regarding data management and improper use that can be made of them. At the same time, however, the great development we have seen in the field of machine learning was made possible also by the free participation, thanks to the presence of free datasets. Moreover some companies could benefit from the use of cloud computing services to obtain predictions on the data they own, but at the same time they do not want to give away the same data. In this project we have followed \cite{dowlin2016cryptonets}, and we have shown how it is possible to formulate a neural network that obtains encrypted predictions starting from input data which are also encrypted, thanks to the use of homomorphic algorithms. We focused on the problems arising from the use of this technology and on how to solve them. Moreover we have seen how the use of the v2.3 version of the SEAL cryptography library leads to a notable slow down in the speed of the algorithm compared to the previous versions, up to 8 times slower. This, however, only in relation to the time of execution: we do not have the necessary skills to comment on the safety of the FV algorithm compared to that used in the previous version, namely the YASHE algorithm, for which the judgment on this topic is suspended, but it is important to note that the choice of the cryptographic scheme also has repercussions on security. The advantage of using this version of SEAL, however, is given by the fact that the numbers that are managed are all with a maximum size of 60 bits, that is they can fit entirely in a register, while in the previous version the values were 192 bits. Although it has not been implemented in our project, this allows an easier implementation of an algorithm that exploits the potential of the GPUs.